\chapter{Introduction}
Compiler optimization refers to the compilation phase that deals with the  generation of better code (code that runs faster and/or takes less space). The information required for doing optimizations are collected through Program analysis. Constant Propagation is an optimization which replaces the use of a variable by a constant, if it can be inferred that every definition of the variable that reaches this use assigns the variable the same constant value. The analysis required for this optimization is the \textit{Reaching Definitions Analysis}. 
\section{Problem Statement}
The problem is to do the Constant Propagation optimization in a program in three address code. This involves the formulation of an analysis to compute the set of variable definitions at every point in the program.
\section{Literature Survey }
The fundamentals of compiler optimization and data-flow analysis are presented by Muchnik  \cite{much}. Gulwani and Necula \cite{sumit} present different techniques for computing equivalent expressions.
\section{Work Done in the previous semester}
Include details of design and implementation done in seventh semester
\section{Work Done in the current semester}
This may consist of implementation and results, if any.
\section{Future Work and Conclusions}
The work that remains to be completed should be stated clearly. 
